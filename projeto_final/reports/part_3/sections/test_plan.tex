%
% test_plan.tex
%
% Copyright (C) 2019 by Gabriel Mariano Marcelino <gabriel.mm8@gmail.com>.
%
% Relatório 3 do Trabalho Final da Disciplina EEL510265.
%
% This work is licensed under the Creative Commons Attribution-ShareAlike 4.0
% International License. To view a copy of this license,
% visit http://creativecommons.org/licenses/by-sa/4.0/.
%

%
% \brief Test plan section.
%
% \author Gabriel Mariano Marcelino <gabriel.mm8@gmail.com>
%
% \version 0.1.0
%
% \date 21/11/2019
%

\section{Plano de Testes} \label{sec:test_plan}

Para testar o correto funcionamento do sistema está se utilizando dois diferentes procedimentos. O primeiro consiste em desenvolver e executar rotinas de testes ao longo de desenvolvimento das classes do programa. Para testar o correto funcionamento das classes e suas respectivas rotinas (métodos), utiliza-se de programas de testes específicos, que testam o comportamento de uma determinada classe de forma isolada. Este procedimento vem sendo adotado desde o início do desenvolvimento do \textit{software}.

Já o segundo tipo de teste, consiste em realizar testes de integração entre diferentes partes do sistema, seja através de rotinas de testes específicas, visando testar a comunicação ou o comportamento de duas ou mais classes funcionando em conjunto (de forma similar ao primeiro tipo de teste apresentado anteriormente), ou ainda através de testes funcionais, neste caso com o sistema funcionando por completo e/ou parcialmente.
