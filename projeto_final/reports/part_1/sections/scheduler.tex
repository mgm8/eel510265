%
% scheduler.tex
%
% Copyright (C) 2019 by Gabriel Mariano Marcelino <gabriel.mm8@gmail.com>.
%
% Relatório 1 do Trabalho Final da Disciplina EEL510265.
%
% This work is licensed under the Creative Commons Attribution-ShareAlike 4.0
% International License. To view a copy of this license,
% visit http://creativecommons.org/licenses/by-sa/4.0/.
%

%
% \brief Scheduler section.
%
% \author Gabriel Mariano Marcelino <gabriel.mm8@gmail.com>
%
% \version 0.1.0
%
% \date 24/10/2019
%

\section{Funcionamento do Escalonador} \label{sec:scheduler}

Para implementar o escalonador de tarefas, pretende-se utilizar o seguinte esquema:

\begin{itemize}
    \item O escalonamento será feito para os métodos das classes (os métodos serão escalonados).
    \item No TCB de cada tarefa serão armazenados o período, a prioridade, o ID e o nome da tarefa.
    \item O tempo será controlado através de um timer (interrupção no caso da plataforma embarcada, ou uma thread no caso da máquina sendo simulada em um sistema operacional).
    \item O escalonamento das tarefas (qual tarefa deve ser executada em um determinado instante de tempo ou \textit{tick} do sistema) deverá ser feito durante a contagem de tempo do \textit{time}, ou seja, nos \textit{ticks} do sistema.
    \item A principio, pretende-se utilizar a politica de escalonamento ``\textit{First come, first served}'', que nada mais eh um que um esquema de FIFO (ou seja, as tarefas sao executadas de acordo com a ordem que chegam). Dependendo dos resultados obtidos, outros algoritmos tambem podem ser testados (EDF, DM, etc.).
\end{itemize}
