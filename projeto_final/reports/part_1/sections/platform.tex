%
% platform.tex
%
% Copyright (C) 2019 by Gabriel Mariano Marcelino <gabriel.mm8@gmail.com>.
%
% Relatório 1 do Trabalho Final da Disciplina EEL510265.
%
% This work is licensed under the Creative Commons Attribution-ShareAlike 4.0
% International License. To view a copy of this license,
% visit http://creativecommons.org/licenses/by-sa/4.0/.
%

%
% \brief Platform section.
%
% \author Gabriel Mariano Marcelino <gabriel.mm8@gmail.com>
%
% \version 0.1.0
%
% \date 24/10/2019
%

\section{Plataforma Alvo} \label{sec:platform}

Para embarcar o \textit{software} desenvolvido, pretende-se utilizar o \textit{kit} de desenvolvimento ``\textit{STM32 F4 Discovery}'' \cite{stm32_discovery} apresentado durante as aulas.

O mesmo é baseado em um microcontrolador ARM Cortex-M4, ou mais precisamente o modelo STM32F429.

Este \textit{kit} de desenvolvimento também possui um \textit{display} de cristal líquido e um botão tátil, que auxiliados por periféricos externos (chaves, botões e LEDs), servirão para a interação entre o sistema e o usuário. Mais especificamente, pretende-se utilizar o \textit{display} para apresentar informações ao usuário (como a opção escolhia, o preço de cada opção e instruções de operação), botões para operar a máquina (botões para escolher o tipo de bebida e o botão ``DEV'' para cancelar a operação escolhida) e simular a entrada de moedas (um botão para cada tipo de moeda aceita). Para se obter as informações de log, pretende-se utilizar uma porta UART, que transmitirá os dados requeridos para um computador quando requisitado.

\subsection{Ferramentas de desenvolvimento}

Para compilar o \textit{software}, tendo o \textit{kit} de desenvolvimento como alvo, se utilizará o compilador GCC para a plataforma ARM (\textit{GNU ARM Embedded Toolchain} \cite{gcc_arm}).

Para carregar o programa compilado no microcontrolador, se utilizará a ferramenta aberta ``\textit{STLink}'' \cite{stlink}.

Como auxílio para o desenvolvimento, também pretende-se utilizar a biblioteca ``\textit{libopencm3}'' \cite{libopencm3}, que permite o acesso aos principais periféricos do microcontrolador e do \textit{kit} de desenvolvimento.
