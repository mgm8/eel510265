%
% messages.tex
%
% Copyright (C) 2019 by Gabriel Mariano Marcelino <gabriel.mm8@gmail.com>.
%
% Relatório 2 do Trabalho Final da Disciplina EEL510265.
%
% This work is licensed under the Creative Commons Attribution-ShareAlike 4.0
% International License. To view a copy of this license,
% visit http://creativecommons.org/licenses/by-sa/4.0/.
%

%
% \brief Messages display model section.
%
% \author Gabriel Mariano Marcelino <gabriel.mm8@gmail.com>
%
% \version 0.1.0
%
% \date 06/11/2019
%

\section{Modelagem do Módulo de Apresentação de Mensagens} \label{sec:messages}

O módulo de apresentação de mensagens será o responsável por receber, armazenar e disponibilizar as mensagens de log do sistema. Conforme especificado, essas mensagens são compostas pelo nome da bebida vendida, o preço da mesma e a data e horário da venda.

Para implementar este módulo, irá se utilizar uma fila implementada através de uma lista duplamente encadeada. Desta forma, tem-se o sistema gerando as mensagens de log em um lado, e as leituras destas mensagens quando requisitado do outro lado da lista. Utilizando-se este tipo de estrutura para esta parte do programa, pode-se acessar o log de mensagens em diferentes tarefas do sistema, como por exemplo, uma tarefa gerando as mensagens quando o evento de venda ocorre, e outra tarefa lendo a fila de mensagens quando a leitura é requisitada.

Para acessar as mensagens geradas durante o funcionamento da máquina, se utilizará uma porta serial (UART) da plataforma embarcada, que enviará as mensagens armazenadas na fila quando um determinado comando é enviado através da mesma. Para a variação do programa que será executada em um microcomputador, as mensagens de log serão gravadas em um arquivo de texto quando requisitado. A relação entre as classes envolvidas no sistema de log pode ser vista na parte inferior da \autoref{fig:class_diagram}.
