%
% scheduler.tex
%
% Copyright (C) 2019 by Gabriel Mariano Marcelino <gabriel.mm8@gmail.com>.
%
% Relatório 2 do Trabalho Final da Disciplina EEL510265.
%
% This work is licensed under the Creative Commons Attribution-ShareAlike 4.0
% International License. To view a copy of this license,
% visit http://creativecommons.org/licenses/by-sa/4.0/.
%

%
% \brief Scheduler model section.
%
% \author Gabriel Mariano Marcelino <gabriel.mm8@gmail.com>
%
% \version 0.1.0
%
% \date 06/11/2019
%

\section{Modelagem da Funcionalidade do Escalonador} \label{sec:scheduler}

Para implementar o escalonador se utilizará a interrupção de um timer da plataforma embarcada para controlar os \textit{ticks} do sistema, ou seja, o controle de tempo. Além disso, essa rotina de interrupção também será utilizada para o controle do escalonamento das tarefas do sistema.

Para armazenar a lista de tarefas a serem executadas pelo sistema, também se utilizará uma fila (implementada com uma fila duplamente encadeada, do mesmo modo que a fila das mensagens de log).

O diagrama de classes da \autoref{fig:class_diagram} não contém as classes relacionadas ao escalonador de tarefas do sistema.
