%
% architecture.tex
%
% Copyright (C) 2019 by Gabriel Mariano Marcelino <gabriel.mm8@gmail.com>.
%
% Relatório final do Trabalho Final da Disciplina EEL510265.
%
% This work is licensed under the Creative Commons Attribution-ShareAlike 4.0
% International License. To view a copy of this license,
% visit http://creativecommons.org/licenses/by-sa/4.0/.
%

%
% \brief System architecture section.
%
% \author Gabriel Mariano Marcelino <gabriel.mm8@gmail.com>
%
% \version 0.1.0
%
% \date 28/11/2019
%

\section{Arquitetura do Sistema} \label{sec:architecture}

\subsection{Organização do Escalonador}

O escalonador do sistema foi implementado na forma de uma classe, contendo todos os parâmetros e métodos necessários para o funcionamento do mesmo. Este utiliza algumas classes como suporte para a sua execução, como a classe ``\textit{Timer}'', que é responsável pelo gerenciamento do contador de \textit{ticks} do sistema (ou seja, o controle de tempo), a classe ``\textit{Task}'' que implementa uma tarefa do sistema (todas as tarefas devem herdar essa classe), e outras classes que implementam estrutura de dados, como as classes ``List'' e ``Queue'' (lista e fila respectivamente).

Para a implementação das tarefas, é necessário implementar as mesmas na forma de uma classe que herda a classe ``\textit{Task}''. Na respectiva classe da tarefa, deve-se implementar os métodos ``\textit{void init()}'' e ``\textit{void run()}'', que são chamados durante a execução do escalonador (o primeiro, uma única vez, e o segundo periodicamente de acordo com os parâmetros de configuração da tarefa, como período e prioridade).

Um exemplo de tarefa implementada deste modo, pode ser visto logo abaixo (tarefa ``\textit{Startup}'' do sistema):

\begin{lstlisting}
/**
 * \brief Startup task.
 */
class TaskStartup: public vmos::Task
{
    public:

        /**
         * \brief Task initialization.
         *
         * \return None.
         */
        void init();

        /**
         * \brief Task implementation.
         *
         * \return None.
         */
        void run();
};
\end{lstlisting}

\subsection{Camadas de Software}

Todo sistema foi organizado e dividido em algumas camadas de abstração. Desta forma, para um melhor entendimento e maior organização, cada camada encontra-se em uma respectiva pasta contendo todos os arquivos fonte referentes a mesma. A divisão de camadas encontra-se listada logo abaixo:

\begin{itemize}
    \item ``\textbf{hal}'' (\textit{Hardware Abstraction Layer}): Arquivos referentes às rotinas de acesso aos periféricos do microcontrolador da plataforma embarcada. Esta camada só é utilizada na versão embarcada do \textit{software}, sendo não utilizada na variação para microcomputadores.
    \item ``\textbf{devices}'': Abstração dos dispositivos e periféricos do sistema (externos ao microcontrolador), com por exemplo o \textit{display} de dados, os botões de comando, etc.
    \item ``\textbf{os}'' (\textit{Operating System}): Arquivos relativos ao escalonador de tarefas e às estrutura de dados (listas, filas, etc.).
    \item ``\textbf{tasks}'': Tarefas dos sistema, a serem executadas pelo escalonador.
\end{itemize}
